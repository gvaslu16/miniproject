\documentclass[12pt, a4paper,oneside]{article}

\usepackage{graphicx}
\graphicspath{ {references/} }

\usepackage[backend=bibtex]{biblatex}
\bibliography{miniproject}
%\bibliographystyle{plain}


\title{Miniproject - CORDIC}
\author{Anders, Gabriel}


\begin{document}

\maketitle

\section{Introduction}
CORDIC is an algorithm that computes trigonometric functions (and many other)
by using simple operations.
%https://en.wikipedia.org/wiki/CORDIC
A typical application would be a two-dimensional vector rotation, as seen in 
figure \ref{fig:two_vector}. Here $(x_{in}, y_{in})$ are the initial coordinates
of the vector, and $(x_{out}, y_{out})$ are the final coordinates.

\begin{figure}[h]
	\centering
	\includegraphics{two_vector.jpg}
	\caption{Two dimensional vector rotation}
	\label{fig:two_vector}
\end{figure}

In order to achieve the operation of rotating the vector, the following equations
have to be calculated:
\[ x_{out} = x_{in} cos\theta - y_{in} sin\theta \]
\[ y_{out} = x_{in} sin\theta + y_{in} cos\theta \]

As seen here, the hardware computing these equations would have to do:
four multiplications, two addition/subtraction 
and access a lookup table for the trigonometric functions\cite{cordic1}.
\\

The idea CORDIC introduces is that we can compute the new coordinates by 
iterating through an algorithm that constantly tries to get closer to the 
final result, using a table containing angles to be used in the next iteration
(micro-rotation). These angles can be either added or subtracted in order
to take the next step, and approximate the target.
\\

The generalized equations of the CORDIC algorithm are:
\[ x_{i+1} = x_i - \sigma_i \cdot 2^{-i} \cdot y_i \]
\[ y_{i+1} = y_i - \sigma_i \cdot 2^{-i} \cdot x_i \]
\[ z_{i+1} = z_i - \sigma_i \cdot arctan(2^{-i}) \]

%https://en.wikibooks.org/wiki/Trigonometry/For_Enthusiasts/The_CORDIC_Algorithm

Here, $x_{i+1}$ and $y_{i+1}$ show us the values respective to each iteration.
In order to know what to do in the next iteration (add/subtract from the previous
angle), we need to compute $\sigma_i$. This is done by looking at the value of 
$z_i$:

$$
\sigma_i = \left\{ \begin{array}{rl}
 +1 &\mbox{ if $z_i>=0$} \\
 -1 &\mbox{ if $z_i<0$}
       \end{array} \right.
$$

$z_i$ keeps track of how much we rotated at every iteration and subtracting that 
from the wanted angle.
The values of $x_i$ and $y_i$ need to be scaled by a factor of $K_i$:

$$K_i = \frac{1}{\sqrt{1 + 2^{-2i}}}$$

However, there are several methods to do this by calculating it in advance or by 
making it a constant. For small architectures, this aspect can be disregarded.
\\
These being said, by looking at the equations we can see that we will need 
to apply the following operations: addition, bitshift and comparison.

\printbibliography

\end{document}